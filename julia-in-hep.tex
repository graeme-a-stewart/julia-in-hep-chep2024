% This paper is Copyright © Graeme Andrew Stewart, Sam Skipsey, 
% authors, 2025.
% Licensed under Creative Commons Attribution 4.0 International (CC BY 4.0), see LICENSE

\documentclass{webofc}
\usepackage{graphicx} % Required for inserting images
\graphicspath{{figures/}}
\usepackage{subcaption}
\usepackage[varg]{txfonts}
\usepackage[utf8]{inputenc}

% Workaround for arXiv, start with "finalizecache" then switch to "frozencache"
% \usepackage[finalizecache,cachedir=.]{minted}
% \usepackage[frozencache,cachedir=.]{minted}

\usepackage{natbib}
\usepackage{hyperref}
\usepackage{enumitem} % This seems to cause nested itemize lists to break...?

%for code snippets
\usepackage{listings}
%%%%%%%%%%%%%%%%%%%%%%%%%%%%%%%%%%%%%%%%%%%%%%%%%%%%%%%%%%%%%%%%%%%%%%%%%%%%%
%% Julia definition (c) 2014 Jubobs
%%
\lstdefinelanguage{Julia}%
  {morekeywords={abstract,begin,break,case,catch,const,continue,do,else,elseif,%
      end,export,false,for,function,immutable,import,importall,if,in,%
      macro,module,otherwise,quote,return,switch,true,try,type,typealias,%
      using,while, where},%
   sensitive=true,%
   alsoother={$},%
   morecomment=[l]\#,%
   morecomment=[n]{\#=}{=\#},%
   morestring=[s]{"}{"},%
   morestring=[m]{'}{'},%
}[keywords,comments,strings]%

\lstset{%
    language         = Julia,
    basicstyle       = \ttfamily,
    keywordstyle     = \textbf,
    %%stringstyle      = \color{magenta},
    commentstyle     = \textit,
    showstringspaces = false,
}
%end for code snippets

% For draftversion
\usepackage{lineno}
\linenumbers

\title{Julia in HEP}

\author{\firstname{Graeme Andrew} \lastname{Stewart}\inst{1}\fnsep\thanks{\email{graeme.andrew.stewart@cern.ch}} \and
\firstname{Sam} \lastname{Skipsey}\inst{2}
% etc.
}

\institute{CERN, Esplanade des Particules 1, Geneva, Switzerland
\and
School of Physics \& Astronomy, University of Glasgow, Glasgow, United Kingdom, G12 8QQ}


\abstract{%
Julia is a mature general-purpose programming language, with a large ecosystem
of libraries and more than 10000 third-party packages, which specifically
targets scientific computing. As a language, Julia is as dynamic, interactive,
and accessible as Python with NumPy, but achieves run-time performance on par
with C/C++. In this paper, we describe the state of adoption of Julia in HEP,
where momentum has been gathering over a number of years.

HEP-oriented Julia packages can, via \texttt{UnROOT.jl}, already read HEP's
major file formats, including TTree and RNTuple formats. Interfaces to some of
HEP's major software packages, such as through \texttt{Geant4.jl}, are available
too. Jet reconstruction algorithms in Julia show excellent performance. A number
of full HEP analyses have been performed in Julia.

We show how, as the support for HEP has matured, developments have benefited
from Julia's core design choices, which makes reuse from and integration with
other packages easy. In particular, libraries developed outside HEP for
plotting, statistics, fitting, and scientific machine learning are extremely
useful.

We believe that the powerful combination of flexibility and speed, the wide
selection of scientific programming tools, and support for all modern
programming paradigms and tools, make Julia the ideal choice for a future
language in HEP.}

\begin{document}

\maketitle

\section{Programming Languages in High-Energy Physics}
\label{sec:introduction}

\subsection{HEP Needs}

High-energy physics (HEP) is a large field, consisting of tens of thousands of
researchers, almost all of whom will need to interact with software and
contribute to software projects during their careers~\cite{2024EPJWC.29505023M}.
It is also one of the biggest, if not the biggest, generators of scientific
datasets today, with exabytes of storage used by the LHC
experiments~\cite{Collaboration:2904204}. This data is processed by a huge
corpus of software, estimated to be many tens of millions of lines in
C++~\cite{hsfcwp}.

This brings a challenge for HEP software. From the point of view of \emph{code
efficiency} we require fast and efficient execution, high throughput, and
scalability at large computer centres and across distributed infrastructures.
Considering \emph{human efficiency} we would like a low barrier to entry for
newcomers, the ability to prototype code rapidly, a broad ecosystem of well
maintained packages, and excellent tooling for developers. These features are
needed to make software able to deal efficiently with huge datasets, as well as
accessible to a large group of developers.

\subsection{From Fortran to the C++/Python Era}

In response to changing technology, and the needs of the field, the programming
languages that are dominant in the field have evolved over time.
From~\cite{pivarski2022} we can identify three major shifts as seen in
\ref{fig:hep-languages}. 

\begin{figure}[htbp]
    \begin{center}
        \includegraphics[width=0.7\textwidth]{hep-programming-languages.png} \\ 
        \caption{Illustrative cartoon of major shifts in HEP programming languages, from~\cite{pivarski2022}.}
        \label{fig:hep-languages}
    \end{center}
\end{figure}

\begin{description}
    \item[From assembler to Fortran] As computers developed from early
    specialised behemoths, with barely an operating system, improved programming
    languages became available. For technical computing Fortran was the most
    effective language and HEP quickly adopted it as it brought a much improved
    syntax, as well as hardware portability.
    \item[From Fortan to C++] Although Fortran had many advantages, HEP had to
    develop language extensions to introduce concepts such as more advanced data
    structures~\cite{Zoll:2296399}. A language which offered native object
    orientation was an attractive choice. In addition, the gap afforded by the
    end of the LEP accelerator and the construction of the LHC allowed the field
    to afford the time for a major language shift.
    \item[The rise of Python] Python earned a well deserved reputation as an
    excellent language for programming efficiency, and also gained ground
    through being the de facto interface to many machine learning libraries. It
    has become widely used in HEP as a compliment to C++.
\end{description}

The current situation for HEP is that C++ and Python are now both widely used,
with each bringing specific advantages, as well as drawbacks. Good C++ excels at
runtime efficiency, but is a difficult language to learn and master as well as
suffering from memory safety issues and being difficult to compose. Python is
expressive, much easier to work with, is safer with memory and composes better
(duck typing). However, it is very slow compared to C++, so not suitable for
high throughput computing.

As noted in~\cite{eschle2023potential}, using two languages is not ideal: it
increases the required expertise, necessitates reimplementation of code for
performance, and reduces code reusability.

\section{Julia}

\subsection{Julia's Motivations}

Some of the design goals of the language. \emph{Julia Programming
Language}~\cite{bib:julia_freshapproach,10.1145/3276490}

The Julia programming language was announced in 2012 with a blog post entitled "Why We Created Julia"~\cite{why-create-julia}.
This document lists a series of ambitious goals for the language, and represents a view into the core developers' mindset formalised in a later paper~\cite{bib:julia_freshapproach,10.1145/3276490}.

Julia provides a syntax as productive as Python, especially for numerical work, whilst leveraging JAOT\footnote{Just-Ahead-Of-Time} compilation to provide speed similar to compiled, statically typed languages like C and Rust. 
It utilises type inference to allow coding in a "gradually typed", generic programming style, although the 
language will always track types for performance behind the scenes, and types can be specified explicitly if required. 
Like MatLab and Fortran, Julia's native operations and type system support arrays as first-class entities, of any dimension, allowing array-oriented code to be both
productively generated, and efficiently executed, with operations naturally "broadcasted" element- or dimension-wise. This allows Julia to also be 
effective for writing linear algebra heavy code. 
In addition, influence from the R community provides a wealth of statistical packages, and an R-flavoured approach to plots and visualisation.
Of course, as with most languages of the 21st century, Julia is a fully open-source language, with its entire codebase freely available (and almost all of it written in Julia itself).

\subsection{Julia in Practice}

Julia's syntax is familiar to programmers conversant with programming performant code in Python with NumPy, except that whitespace is not
syntactically relevant (blocks end with \textit{end}).

For example, listing~\ref{listing1} shows some toy code to generate a grayscale image of the famous Mandelbr\"{o}t set:
\begin{lstlisting}[language=Julia, caption={}, label=listing1]  
using Images

function mandel(z)
    c = z
    maxiter = 80
    for n = 1:maxiter
        if abs2(z) > 4
            return n-1
        end
        z = z^2 + c
    end
    maxiter
end

set = [ mandel(complex(r,i)) for i=-1.:.01:1., r=-2.0:.01:0.5 ]
img = Gray.(set ./ 80)
\end{lstlisting}

Here we demonstrate importing of packages (line 1), function declaration with implicit types and explicit loop and branch constructs, implicit loops via ``comprehensions'' to generate a  value for every point in an implicitly defined array, and, finally, transparent broadcasting of operations over that array (both the function call to Gray, and the division operation are distributed over the whole array). 
Unlike Python, explicit loops are optimised efficiently, and are no slower than comprehensions or "functional-style" iterators (which are also supported by Julia).



%Tooling and ecosystem.
As with other languages of the 2010s, Julia comes with a set of tools for managing development environments beyond the interpreter.

`juliaup` allows seamless management of multiple Julia releases on the same machine, including tracking particular patch releases, and 
choosing the system default.

The integrated package management in Julia - via the Pkg library, or a sub-environment within a Julia REPL accessed by pressing '$]$', 
tracks and maintains the dependency graph of a Julia project. State is entirely stored within two human-readable files - Project.toml (which represents the 
direct dependencies of the project) and Manifest.toml (which preserves the exact resulting environment, including secondary dependencies and exact versions).
. The user of a Julia codebase can easily reproduce the exact environment used by the codebase as long as these files are provided.

The Julia Package "General Registry" indexes packages and their releases, and (as with other languages such as Rust and Javascript) relies on a public
index hosted on GitHub (with off-GitHub backups).
A help environment, accessed by "?" on an empty line of the REPL, allows interactive help on any keyword or symbol known to the REPL; it is trivial to add documentation to a function or type 
by simply prepending its definition with a triple-quoted string - this is immediately available to the help environment.

Leveraging the fact that Julia is JAOT-compiled, and that this allows its entire standard library to be written in idiomatic Julia, the REPL also provides
a series of powerful macro utilities for inspecting the byte- and machine-code generated for a given expression (\verb$@code_lowered$, \verb$@code_native$) and
for locating and displaying the source code for any function in the current namespace, including from the standard library (\verb$@less$), as well as other introspection
tools. Profiling of code in the REPL is similarly directly supported via macros (\verb$@benchmark$...), which provide detailed performance sampling, including execution time distributions and GC invocations.
Extended introspection and profiling tools are also available in optional packages, such as \texttt{About.jl}, which can provide information 
on memory layout of datatypes and thread safety of functions.

A well-supported VSCode extension is available for the language, which also supports the standard LSP allowing it to support development in other editors. This includes 
the usual benefits of code completion (and Unicode completion for non-ASCII characters), linting, highlighting and so on.

Finally, Julia is one of the founding languages supported by Jupyter - being the "Ju" in the portmanteau - and also provides other native notebook implementations such as Pluto.jl. 

\subsection{Key Design Features for Performance}

Type system.

Multiple dispatch.

\section{Julia for Scientific Computing}

General adoption:~\cite{perkel-julia-science}.

GPU programming.

Some HPC codes.

\section{Julia in HEP}

\subsection{Challenges}

What does HEP need from its computing?

Cite some general overviews:~\cite{Stanitzki:2020bnx,eschle2023potential}.

\subsection{HEP Data Formats}

We can read that data... UpROOT.jl. EDM4hep.jl.

\subsection{Event Generators}

A bit about QuantumElectrodynamics.jl.

\subsection{Simulation}

Detector simulation is a crucial component of every High Energy Physics (HEP) experiment, playing a key role both during the design and conception of the detector and later in data analysis. The most widely used toolkit for this purpose is Geant4 \cite{GEANT4:2002zbu}, a C++-based framework with over 2 million lines of code.

Given its complexity and extensive adoption, a complete rewrite of Geant4 in a new language is highly impractical. Instead, this presents an opportunity to explore Julia's interoperability with other languages. One particular challenge arises from Geant4's callback-based user interface, which relies on C++ virtual methods invoked at specific points during particle transport. Application developers must implement these callbacks to configure and control the simulation and extract relevant simulated data. However, integrating this mechanism in Julia is more complex than in other languages, as Julia does not natively support virtual methods.

The Geant4.jl \cite{geant4-jl-github} package has been developed to provide a Julia interface to Geant4. It leverages CxxWrap.jl \cite{cxxwrap-jl-github}, a package that enables calling C++ functions and types from Julia. Similar to Python’s static bindings, invoking C++ code from Julia requires explicit wrapper definitions for each method exposed to Julia. However, given Geant4's large and complex codebase, manually writing and maintaining these wrappers is not a viable approach, especially for making it more sustainable with future toolkit updates. To address this, we use WrapIt \cite{wrapit-github}, a package that automates wrapper generation by utilizing the Clang library to parse C++ header files and extract class declarations. This automation significantly reduces development effort and ensures long-term maintainability of the interface.

Integrating Geant4 with Julia allows researchers to take advantage of Julia's high-level programming capabilities and performance benefits while retaining the full functionality and efficiency of the Geant4 toolkit. This integration also provided an opportunity to rethink and streamline the interface, making it more intuitive and user-friendly. In particular, we focused on ensuring that application developers can concentrate on the essential aspects of their simulations while minimizing configuration overhead. Boilerplate code and C++ idiosyncrasies are hidden, allowing for a cleaner, more concise approach to defining simulations. The performance of the Geant4.jl package is comparable to that of the native C++ Geant4 toolkit, demonstrating the feasibility of using Julia for HEP detector simulation.

\subsection{Reconstruction}

JetReconstruction.jl

\subsection{Analysis}

Overview of analysis papers and suitability of Julia.

\subsection{End-to-end Computing}

The Legend Julia stack.

\section{Conclusions}

It's all good, nothing can go wrong. To infinity and beyond, etc.

\sloppy
\raggedright
% \clearpage
\bibliography{julia-in-hep}


\end{document}
