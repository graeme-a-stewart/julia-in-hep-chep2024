% This paper is Copyright © Graeme Andrew Stewart and other
% authors, 2025.
% Licensed under Creative Commons Attribution 4.0 International (CC BY 4.0), see LICENSE

\documentclass{webofc}
\usepackage{graphicx} % Required for inserting images
\graphicspath{{figures/}}
\usepackage{subcaption}
\usepackage[varg]{txfonts}
\usepackage[utf8]{inputenc}

% Workaround for arXiv, start with "finalizecache" then switch to "frozencache"
% \usepackage[finalizecache,cachedir=.]{minted}
% \usepackage[frozencache,cachedir=.]{minted}

\usepackage{natbib}
\usepackage{hyperref}
\usepackage{enumitem} % This seems to cause nested itemize lists to break...?

% For draft version
\usepackage{lineno}
\linenumbers

\title{Julia in HEP}

\author{\firstname{Graeme Andrew} \lastname{Stewart}\inst{1}\fnsep\thanks{\email{graeme.andrew.stewart@cern.ch}} \and
\firstname{Other} \lastname{People}\inst{2}
% etc.
}

\institute{CERN, Esplanade des Particules 1, Geneva, Switzerland
\and
Some Other Place
}

\abstract{%
Julia is a mature general-purpose programming language, with a large ecosystem
of libraries and more than 10000 third-party packages, which specifically
targets scientific computing. As a language, Julia is as dynamic, interactive,
and accessible as Python with NumPy, but achieves run-time performance on par
with C/C++. In this paper, we describe the state of adoption of Julia in HEP,
where momentum has been gathering over a number of years.

HEP-oriented Julia packages can, via \texttt{UnROOT.jl}, already read HEP's
major file formats, including TTree and RNTuple formats. Interfaces to some of
HEP's major software packages, such as through \texttt{Geant4.jl}, are available
too. Jet reconstruction algorithms in Julia show excellent performance. A number
of full HEP analyses have been performed in Julia.

We show how, as the support for HEP has matured, developments have benefited
from Julia's core design choices, which makes reuse from and integration with
other packages easy. In particular, libraries developed outside HEP for
plotting, statistics, fitting, and scientific machine learning are extremely
useful.

We believe that the powerful combination of flexibility and speed, the wide
selection of scientific programming tools, and support for all modern
programming paradigms and tools, make Julia the ideal choice for a future
language in HEP.}

\begin{document}

\maketitle

\section{Programming Languages in High-Energy Physics}
\label{sec:introduction}

\subsection{HEP Needs}

High-energy physics (HEP) is a large field, consisting of tens of thousands of
researchers, almost all of whom will need to interact with software and
contribute to software projects during their careers~\cite{2024EPJWC.29505023M}.
It is also one of the biggest, if not the biggest, generators of scientific
datasets today, with exabytes of storage used by the LHC
experiments~\cite{Collaboration:2904204}. This data is processed by a huge
corpus of software, estimated to be many tens of millions of lines in
C++~\cite{hsfcwp}.

This brings a challenge for HEP software. From the point of view of \emph{code
efficiency} we require fast and efficient execution, high throughput, and
scalability at large computer centres and across distributed infrastructures.
Considering \emph{human efficiency} we would like a low barrier to entry for
newcomers, the ability to prototype code rapidly, a broad ecosystem of well
maintained packages, and excellent tooling for developers. These features are
needed to make software able to deal efficiently with huge datasets, as well as
accessible to a large group of developers.

\subsection{From Fortran to the C++/Python Era}

In response to changing technology, and the needs of the field, the programming
languages that are dominant in the field have evolved over time.
From~\cite{pivarski2022} we can identify three major shifts as seen in
\ref{fig:hep-languages}. 

\begin{figure}[htbp]
    \begin{center}
        \includegraphics[width=0.7\textwidth]{hep-programming-languages.png} \\ 
        \caption{Illustrative cartoon of major shifts in HEP programming languages, from~\cite{pivarski2022}.}
        \label{fig:hep-languages}
    \end{center}
\end{figure}

\begin{description}
    \item[From assembler to Fortran] As computers developed from early
    specialised behemoths, with barely an operating system, improved programming
    languages became available. For technical computing Fortran was the most
    effective language and HEP quickly adopted it as it brought a much improved
    syntax, as well as hardware portability.
    \item[From Fortan to C++] Although Fortran had many advantages, HEP had to
    develop language extensions to introduce concepts such as more advanced data
    structures~\cite{Zoll:2296399}. A language which offered native object
    orientation was an attractive choice. In addition, the gap afforded by the
    end of the LEP accelerator and the construction of the LHC allowed the field
    to afford the time for a major language shift.
    \item[The rise of Python] Python earned a well deserved reputation as an
    excellent language for programming efficiency, and also gained ground
    through being the de facto interface to many machine learning libraries. It
    has become widely used in HEP as a compliment to C++.
\end{description}

The current situation for HEP is that C++ and Python are now both widely used,
with each bringing specific advantages, as well as drawbacks. Good C++ excels at
runtime efficiency, but is a difficult language to learn and master as well as
suffering from memory safety issues and being difficult to compose. Python is
expressive, much easier to work with, is safer with memory and composes better
(duck typing). However, it is very slow compared to C++, so not suitable for
high throughput computing.

As noted in~\cite{eschle2023potential}, using two languages is not ideal: it
increases the required expertise, necessitates reimplementation of code for
performance, and reduces code reusability.

\section{Julia}

\subsection{Julia's Motivations}

Some of the design goals of the language. \emph{Julia Programming
Language}~\cite{bib:julia_freshapproach,10.1145/3276490}

\subsection{Julia in Practice}

Some code examples - demonstrate \emph{ease} and also give some benchmarks for
speed.

Tooling and ecosystem.

\subsection{Key Design Features for Performance}

Type system.

Multiple dispatch.

\section{Julia for Scientific Computing}

General adoption:~\cite{perkel-julia-science}.

GPU programming.

Some HPC codes.

\section{Julia in HEP}

\subsection{Challenges}

What does HEP need from its computing?

Cite some general overviews:~\cite{Stanitzki:2020bnx,eschle2023potential}.

\subsection{HEP Data Formats}

We can read that data... UpROOT.jl. EDM4hep.jl.

\subsection{Event Generators}

The numerical calculation of amplitudes for hard scattering processes and Monte Carlo event
generation plays a fundamental role in high-energy physics analysis workflows, with a
long history (\cite{campbell2024event}). Consequently, integrating various aspects of physics models with specialized
numerical algorithms in a high-performance computing environment is recognized as one major challenge
in HEP software development for future HEP experiments (\cite{HEPSoftwareFoundation:2017ggl, HSFPhysicsEventGeneratorWG:2020gxw, HSFPhysicsEventGeneratorWG:2021xti}) and beyond.

With the open-source framework \texttt{QuantumElectrodynamics.jl} \cite{qedjl-github}, we take the first steps in exploring
how the Julia programming language can address these challenges. Specifically, our framework
facilitates the numerical calculation of scattering amplitudes and the implementation of Monte
Carlo event generation within the domain of perturbative and strong-field quantum electrodynamics (cf. \cite{Fedotov:2022ely}).
The overall structure of the framework is unified through interfaces defined in \texttt{QEDbase.jl}, which provides
standardized representations for fundamental mathematical objects such as four-momenta, bi-spinors, and
phase space points. Additionally, these interfaces extend to the configuration entities like scattering processes,
computational models, and phase space layouts.
Further interfaces are provided for computable quantities, such as differential cross-sections,
and various samplers for Monte Carlo event generation. Here, Julia’s multiple dispatch mechanism
proves particularly powerful, allowing different implementations to be seamlessly integrated without explicitly
specifying types within the interface definitions. Moreover, the domain-specific language facilitated by
these interfaces—covering elements such as processes, models, and phase space layouts—combined with
multiple dispatch enables the straightforward incorporation of analytical formulas whenever they are available.
This typically reduces to implementing a single method for a specific function signature.

Beyond its interface definitions, \texttt{QuantumElectrodynamics.jl} provides concrete implementations for all
major components: \texttt{QEDcore.jl} handles the fundamental mathematical structures, \texttt{QEDprocesses.jl} computes
differential cross sections, and \texttt{QEDevents.jl} offers different samplers for event generation.
The \texttt{QEDFeynmanDiagrams.jl} package, as part of \texttt{QEDprocesses.jl}, facilitates the calculation of
scattering amplitudes for arbitrary QED processes by leveraging Julia’s metaprogramming and code generation
capabilities. Built on top of \texttt{ComputableDAGs.jl}, the generated code can be directly analyzed and manipulated
within Julia itself, even without leaving the session. This enables meta-optimizations based on domain-specific knowledge, such as recognizing patterns in Feynman diagrams and exploiting mathematical properties like distributivity.

Finally, by leveraging multiple dispatch and various array abstractions from the Julia ecosystem—such as \texttt{CUDA.jl}
and \texttt{AMDGPU.jl}—the generated code can be seamlessly compiled and executed on GPUs in addition to CPUs, without
requiring any modifications to its underlying structure.
While some framework components are still under development and will be further extended to address
broader challenges in high-energy physics, the initial structures demonstrate great potential. The modular design,
combined with Julia’s capabilities, has the potential to contribute to future developments in numerical calculations
and event generation within HEP.


\subsection{Simulation}

Detector simulation is a crucial component of every High Energy Physics (HEP) experiment, playing a key role both during the design and conception of the detector and later in data analysis. The most widely used toolkit for this purpose is Geant4 \cite{GEANT4:2002zbu}, a C++-based framework with over 2 million lines of code.

Given its complexity and extensive adoption, a complete rewrite of Geant4 in a new language is highly impractical. Instead, this presents an opportunity to explore Julia's interoperability with other languages. One particular challenge arises from Geant4's callback-based user interface, which relies on C++ virtual methods invoked at specific points during particle transport. Application developers must implement these callbacks to configure and control the simulation and extract relevant simulated data. However, integrating this mechanism in Julia is more complex than in other languages, as Julia does not natively support virtual methods.

The Geant4.jl \cite{geant4-jl-github} package has been developed to provide a Julia interface to Geant4. It leverages CxxWrap.jl \cite{cxxwrap-jl-github}, a package that enables calling C++ functions and types from Julia. Similar to Python’s static bindings, invoking C++ code from Julia requires explicit wrapper definitions for each method exposed to Julia. However, given Geant4's large and complex codebase, manually writing and maintaining these wrappers is not a viable approach, especially for making it more sustainable with future toolkit updates. To address this, we use WrapIt \cite{wrapit-github}, a package that automates wrapper generation by utilizing the Clang library to parse C++ header files and extract class declarations. This automation significantly reduces development effort and ensures long-term maintainability of the interface.

Integrating Geant4 with Julia allows researchers to take advantage of Julia's high-level programming capabilities and performance benefits while retaining the full functionality and efficiency of the Geant4 toolkit. This integration also provided an opportunity to rethink and streamline the interface, making it more intuitive and user-friendly. In particular, we focused on ensuring that application developers can concentrate on the essential aspects of their simulations while minimizing configuration overhead. Boilerplate code and C++ idiosyncrasies are hidden, allowing for a cleaner, more concise approach to defining simulations. The performance of the Geant4.jl package is comparable to that of the native C++ Geant4 toolkit, demonstrating the feasibility of using Julia for HEP detector simulation.

\subsection{Reconstruction}

JetReconstruction.jl

\subsection{Analysis}

Overview of analysis papers and suitability of Julia.

\subsection{End-to-end Computing}

The LEGEND~\cite{LEGEND:2017AIPC} experiment demonstrates how Julia can be used as a basis for end-to-end analysis in a larger physics experiment with multiple subsystems. LEGEND uses Julia as its official secondary software stack, both to verify the results of the primary software stack (written in Python) and as a test-bed for future software technologies. The whole data analysis chain, encompassing raw waveform data signal processing, ML-based data quality cuts, data calibration, event building and high-level statistical analysis is implemented completely in Julia. In addition, LEGEND uses the Bayesian Analysis Toolkit in Julia (BAT.jl)~\cite{Schulz:2021BAT} as it's primary Bayesian framework for both background decomposition and final physics analysis, and the Julia package SolidStateDetectors.jl~\cite{Abt:2021SSD} for detector simulation and detector design. With the exception of a custom Geant4-based software, the LEGEND collaboration is now able to perform any simulation and analysis task in Julia.


\section{Conclusions}

It's all good, nothing can go wrong. To infinity and beyond, etc.

\sloppy
\raggedright
% \clearpage
\bibliography{julia-in-hep}

\end{document}
